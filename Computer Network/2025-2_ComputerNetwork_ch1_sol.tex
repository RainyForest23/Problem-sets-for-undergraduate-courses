\documentclass[a4paper, 10pt]{article}

% pdfLaTeX용 한글 패키지
\usepackage{kotex}

% 기본 패키지
\usepackage{amsmath, amsfonts, amssymb}
\usepackage{graphicx}
\usepackage{geometry}
\usepackage{fancyhdr}
\usepackage{enumitem}
\usepackage{array}
\usepackage{booktabs}
\usepackage{xcolor}
\usepackage{tikz}
\usepackage[colorlinks=true, urlcolor=blue, linkcolor=blue]{hyperref} % 하이퍼링크 (가장 마지막에)

% 페이지 여백 설정
\geometry{
  top=2.5cm, 
  bottom=2.5cm, 
  left=2cm, 
  right=2cm,
  headheight=20pt
}

% 머리글 및 바닥글 설정 (해설지용으로 수정)
\pagestyle{fancy}
\fancyhf{}
\renewcommand{\headrulewidth}{0.8pt}
\renewcommand{\footrulewidth}{0.4pt}
\lhead{\small 2025학년도 중간고사 대비 연습문제}
\chead{\small\textbf{컴퓨터 네트워크 - 정답 및 해설}}
\rhead{\small\thepage}
\cfoot{\scriptsize ※ 본 해설지는 학습 목적으로 제작되었습니다.}

% 원형 숫자
\newcommand*\circled[1]{\tikz[baseline=(char.base)]{
  \node[shape=circle,draw,inner sep=0.7pt,minimum size=0.7em,line width=0.5pt] (char) {\small #1};}}

% 객관식 문항 스타일
\newlist{mchoice}{enumerate}{1}
\setlist[mchoice,1]{
  label=\protect\circled{\arabic*},
  leftmargin=2em,
  itemsep=0.3em,
  parsep=0pt
}

% 커스텀 명령어
\newcommand{\sectionbox}[1]{
  \vspace{0.5em}
  \noindent\fbox{\textbf{#1}}
  \vspace{0.5em}
}

\setlength{\parindent}{0pt}

\begin{document}

\begin{center}
  \Large\textbf{2025학년도 컴퓨터네트워크 midterm 대비 연습문제 정답 및 해설}\\[8pt]
  \large\textbf{Chapter 1. Introduction}\\[5pt]
  \small\textbf{contact: \href{mailto:wrim0923@ewhain.net}{wrim0923@ewhain.net} | github: \href{https://github.com/RainyForest23}{RainyForest23}}\\[5pt]
  \small\textcolor{gray}{Last updated: \today}
\end{center}

\vspace{1cm}

% --- 빠른 정답 ---
\section*{빠른 정답}
\noindent\rule{\textwidth}{0.6pt}
\begin{center}
\renewcommand{\arraystretch}{1.4} % 표의 줄 간격 조절
\begin{tabular}{*{5}{p{0.19\textwidth}}}
\toprule
\textbf{1.} \circled{1} & \textbf{2.} 호스트, 엣지 & \textbf{3.} ISP & \textbf{4.} RFC, IETF & \textbf{5.} 라우터, 스위치 \\
\textbf{6.} 참 & \textbf{7.} HFC & \textbf{8.} \circled{3} & \textbf{9.} AON, PON & \textbf{10.} 5G / 셀룰러 \\
\textbf{11.} 예약, 전용, 보장 & \textbf{12.} 통계적 다중화 & \textbf{13.} 포워딩, 라우팅 & \textbf{14.} 전체 패킷 & \textbf{15.} 주파수, 시간 \\
\textbf{16.} (해설 참조) & \textbf{17.} L/R, d/s & \textbf{18.} 1 & \textbf{19.} 유한 & \textbf{20.} 병목 \\
\textbf{21.} 피어링, 아니요 & \textbf{22.} IXP & \textbf{23.} Tier-1 & \textbf{24.} 멀티호밍 & \textbf{25.} (해설 참조) \\
\textbf{26.} \circled{4} & \textbf{27.} (해설 참조) & \textbf{28.} 바이러스 & \textbf{29.} 송신지 IP & \textbf{30.} Processing \\
\bottomrule
\end{tabular}
\end{center}

\newpage

% --- 상세 해설 ---
\section*{상세 해설}
\noindent\rule{\textwidth}{0.8pt}

\begin{enumerate}[itemsep=2.5em, leftmargin=2em, label={}]

\item[] \sectionbox{I. 용어 및 기본 개념 (5문제)}

\item[\textbf{1.}] ...
\vspace{0.5em}
\noindent\textbf{정답: \circled{1}} \par
\small\textbf{해설:} 프로토콜은 통신하는 개체들 간의 메시지 형식(syntax), 순서(order), 그리고 메시지 송수신 시 취해야 할 행동(action)을 정의하는 규칙의 집합입니다.

\item[\textbf{2.}] ...
\vspace{0.5em}
\noindent\textbf{정답:} a. 호스트(Host) b. 엣지(Edge) \par
\small\textbf{해설:} 종단 시스템은 인터넷에 연결된 컴퓨터 및 기타 장치들을 의미하며, 호스트라고도 부릅니다. 이들은 인터넷의 가장자리, 즉 '엣지'에 위치하여 애플리케이션을 실행합니다.

\item[\textbf{3.}] ...
\vspace{0.5em}
\noindent\textbf{정답:} a. ISP (Internet Service Providers) \par
\small\textbf{해설:} 인터넷은 단일 네트워크가 아니라, 전 세계 수많은 ISP들이 계층적으로 또는 수평적으로 서로 연결되어 구성된 거대한 '네트워크의 집합체'입니다.

\item[\textbf{4.}] ...
\vspace{0.5em}
\noindent\textbf{정답:} a. RFC (Request for Comments) b. IETF (Internet Engineering Task Force) \par
\small\textbf{해설:} IETF는 인터넷의 주요 프로토콜(TCP, IP, HTTP 등)을 개발하는 핵심 기관이며, 이들의 표준 문서를 RFC라고 합니다.

\item[\textbf{5.}] ...
\vspace{0.5em}
\noindent\textbf{정답:} 라우터(Routers), 링크 계층 스위치(Link-layer switches) \par
\small\textbf{해설:} 라우터와 링크 계층 스위치는 모두 패킷을 전달하는 핵심적인 패킷 스위치 장비이지만, 각각 네트워크 계층(3계층)과 링크 계층(2계층)에서 동작한다는 차이가 있습니다.

\item[] \sectionbox{II. 네트워크 엣지 및 접속 네트워크 (5문제)}

\item[\textbf{6.}] ...
\vspace{0.5em}
\noindent\textbf{정답: 참} \par
\small\textbf{해설:} 제시된 문장은 액세스 네트워크의 정확한 정의입니다. 따라서 '거짓'인 설명을 고르는 문제의 답이 될 수 없습니다. (문제 의도상 해당 문장의 참/거짓을 판별)

\item[\textbf{7.}] ...
\vspace{0.5em}
\noindent\textbf{정답:} a. HFC (Hybrid Fiber Coax) \par
\small\textbf{해설:} HFC는 광섬유와 동축 케이블을 함께 사용하는 방식으로, 기존 케이블 TV 망을 통해 인터넷 서비스를 제공하는 대표적인 주거용 액세스 기술입니다.

\item[\textbf{8.}] ...
\vspace{0.5em}
\noindent\textbf{정답: \circled{3}} \par
\small\textbf{해설:} DSL은 각 가정이 전화국의 중앙 오피스(CO)까지 기존 전화선을 통해 '전용' 회선을 사용합니다. 반면 HFC는 케이블 헤드엔드로부터 나온 회선을 여러 가정이 '공유'하여 사용합니다.

\item[\textbf{9.}] ...
\vspace{0.5em}
\noindent\textbf{정답:} a. AON (Active Optical Network) b. PON (Passive Optical Network) \par
\small\textbf{해설:} AON은 각 가정에 전용 광섬유를 제공하는 스위치드 이더넷 방식이며, PON은 광 분배기(splitter)를 사용하여 하나의 광섬유를 여러 가정이 공유하는 수동형 방식입니다.

\item[\textbf{10.}] ...
\vspace{0.5em}
\noindent\textbf{정답:} a. 5G / 셀룰러 기술 \par
\small\textbf{해설:} FWA는 5G와 같은 무선 이동통신 기술을 사용하여 유선과 유사한 초고속 인터넷을 '고정된' 장소(가정, 사무실)에 제공하는 기술입니다.

\item[] \sectionbox{III. 네트워크 코어 및 스위칭 (5문제)}

\item[\textbf{11.}] ...
\vspace{0.5em}
\noindent\textbf{정답:} a. 예약(reservation) b. 전용(dedicated) c. 보장된(guaranteed) \par
\small\textbf{해설:} 서킷 스위칭은 통신 시작 전에 경로상의 자원(대역폭, 스위치 용량)을 미리 예약하고, 통신이 끝날 때까지 해당 자원을 독점적으로 사용합니다. 이로 인해 일정한 성능이 보장됩니다.

\item[\textbf{12.}] ...
\vspace{0.5em}
\noindent\textbf{정답:} a. 통계적 다중화 (Statistical Multiplexing) \par
\small\textbf{해설:} 패킷 스위칭은 자원을 미리 예약하지 않고 필요할 때만 공유하여 사용합니다. 모든 사용자가 항상 데이터를 보내는 것은 아니므로, 이러한 'on-demand' 자원 공유를 통해 더 많은 사용자를 수용할 수 있습니다.

\item[\textbf{13.}] ...
\vspace{0.5em}
\noindent\textbf{정답:} a. 포워딩(Forwarding) b. 라우팅(Routing) \par
\small\textbf{해설:} 포워딩은 라우터 내에서 패킷을 입력 포트에서 출력 포트로 전달하는 단순한 동작입니다. 라우팅은 라우팅 알고리즘을 통해 패킷의 전체 경로를 결정하는 더 복잡하고 지능적인 과정입니다.

\item[\textbf{14.}] ...
\vspace{0.5em}
\noindent\textbf{정답:} a. 전체 패킷(the entire packet) \par
\small\textbf{해설:} 라우터는 패킷의 첫 비트가 도착하자마자 전달을 시작하는 것이 아니라, L-bit 크기의 패킷 전체를 수신(저장)한 후에야 비로소 다음 링크로 전달(전송)을 시작합니다.

\item[\textbf{15.}] ...
\vspace{0.5em}
\noindent\textbf{정답:} a. 주파수 스펙트럼(frequency spectrum), 시간(time) \par
\small\textbf{해설:} FDM은 주파수라는 자원을, TDM은 시간이라는 자원을 잘게 쪼개어 각 사용자(통화)에게 할당하는 다중화 방식입니다.

\item[] \sectionbox{IV. 성능: 지연, 손실, 처리량 (5문제)}

\item[\textbf{16.}] ...
\vspace{0.5em}
\noindent\textbf{정답:} Processing, Queueing, Transmission, Propagation \par
\small\textbf{해설:} 패킷이 라우터를 거칠 때 발생하는 4대 지연 요소는 처리, 큐잉, 전송, 전파 지연입니다.

\item[\textbf{17.}] ...
\vspace{0.5em}
\noindent\textbf{정답:} a. L/R  b. d/s \par
\small\textbf{해설:} 전송 지연은 패킷의 모든 비트를 링크에 밀어넣는 데 걸리는 시간(패킷 크기/전송률)이며, 전파 지연은 한 비트가 링크의 한쪽 끝에서 다른 쪽 끝까지 이동하는 데 걸리는 시간(거리/속도)입니다.

\item[\textbf{18.}] ...
\vspace{0.5em}
\noindent\textbf{정답:} a. 1 \par
\small\textbf{해설:} 트래픽 강도(La/R)는 링크에 도착하는 트래픽의 양이 링크가 처리할 수 있는 용량에 얼마나 가까운지를 나타냅니다. 이 값이 1에 가까워지면(즉, 큐가 거의 항상 차 있으면) 큐 대기 시간이 무한대에 가깝게 급증합니다.

\item[\textbf{19.}] ...
\vspace{0.5em}
\noindent\textbf{정답:} a. 유한(finite) \par
\small\textbf{해설:} 라우터의 큐(버퍼)는 메모리 크기가 정해져 있어 용량이 유한합니다. 패킷이 도착했을 때 이 큐가 꽉 차 있으면, 해당 패킷은 버려져 손실이 발생합니다.

\item[\textbf{20.}] ...
\vspace{0.5em}
\noindent\textbf{정답:} b. 병목(Bottleneck) 링크 \par
\small\textbf{해설:} 종단 간 경로를 구성하는 여러 링크들 중 전송률(대역폭)이 가장 낮은 링크가 전체 처리량을 결정하게 되는데, 이를 병목 링크라고 합니다.

\item[] \sectionbox{V. 인터넷 구조 및 계층 (5문제)}

\item[\textbf{21.}] ...
\vspace{0.5em}
\noindent\textbf{정답:} a. 인터넷 피어링(Internet Peering) b. 아니요(No) \par
\small\textbf{해설:} 동일 계층의 대형 ISP들은 서로의 네트워크에 있는 고객들에게 도달하기 위해 중개 ISP 없이 직접 연결하는데, 이를 피어링이라고 합니다. 보통 상호 이익을 위해 비용을 정산하지 않습니다(settlement-free).

\item[\textbf{22.}] ...
\vspace{0.5em}
\noindent\textbf{정답:} a. 인터넷 교환 지점 (Internet Exchange Point) \par
\small\textbf{해설:} IXP는 여러 ISP들이 라우터를 가져와 서로 효율적으로 트래픽을 교환할 수 있도록 중립적인 물리적 장소를 제공하는 시설입니다.

\item[\textbf{23.}] ...
\vspace{0.5em}
\noindent\textbf{정답:} a. Tier-1 \par
\small\textbf{해설:} Tier-1 ISP는 전 세계 인터넷의 최상위 계층을 형성하는 소수의 대규모 ISP로, 다른 ISP에게서 인터넷 연결을 구매할 필요가 없습니다.

\item[\textbf{24.}] ...
\vspace{0.5em}
\noindent\textbf{정답:} a. 멀티호밍 (Multi-homing) \par
\small\textbf{해설:} 하나의 ISP가 안정성과 경로 최적화를 위해 두 개 이상의 상위(Provider) ISP에 동시에 연결하는 것을 멀티호밍이라고 합니다.

\item[\textbf{25.}] ...
\vspace{0.5em}
\noindent\textbf{정답:} 5. Application (HTTP), 4. Transport (TCP), 3. Network (IP), 2. Link (Ethernet) \par
\small\textbf{해설:} 인터넷 5계층 모델은 애플리케이션, 트랜스포트, 네트워크, 링크, 피지컬 계층으로 구성됩니다.

\item[] \sectionbox{VI. 보안 및 기타 (5문제)}

\item[\textbf{26.}] ...
\vspace{0.5em}
\noindent\textbf{정답: \circled{4}} \par
\small\textbf{해설:} 인터넷의 핵심 설계 철학 중 하나는 '최선형 노력(Best-effort)' 서비스입니다. 즉, 네트워크는 패킷 전송을 보장하지 않으며, 엄격한 QoS 보장은 IP 계층의 역할이 아닙니다.

\item[\textbf{27.}] ...
\vspace{0.5em}
\noindent\textbf{정답:} a. 서버(server), 링크(link) b. 1. 타겟(target), 3. 타겟(target) \par
\small\textbf{해설:} DoS 공격은 대량의 위조 트래픽으로 목표 서버나 네트워크 링크를 마비시켜 정상적인 사용을 방해하는 것을 목표로 합니다.

\item[\textbf{28.}] ...
\vspace{0.5em}
\noindent\textbf{정답:} a. 바이러스(Virus) \par
\small\textbf{해설:} 바이러스는 스스로 복제하여 전파되지 않고, 사용자가 감염된 파일을 실행하는 등의 개입을 통해 다른 프로그램에 기생하여 퍼져나갑니다. (스스로 전파되면 웜(Worm)입니다.)

\item[\textbf{29.}] ...
\vspace{0.5em}
\noindent\textbf{정답:} a. 송신지 IP (source IP) \par
\small\textbf{해설:} IP 스푸핑은 패킷의 발신지 IP 주소를 다른 주소로 위조하여, 수신자가 패킷의 출처를 속이거나 방화벽을 우회하려는 목적으로 사용되는 공격 기법입니다.

\item[\textbf{30.}] ...
\vspace{0.5em}
\noindent\textbf{정답:} a. 처리(Processing) delay \par
\small\textbf{해설:} 처리 지연은 라우터가 패킷 헤더의 비트 오류를 검사하고, 목적지 주소를 확인하여 어떤 출력 포트로 보낼지 결정하는 데 걸리는 시간입니다.

\end{enumerate}

\vfill
\begin{center}
  \rule{0.9\textwidth}{0.4pt}\\[8pt]
  {\small\textbf{--- 수고하셨습니다 ---}}\\[6pt]
\end{center}

\end{document}