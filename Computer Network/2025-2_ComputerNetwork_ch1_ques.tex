\documentclass[a4paper, 10pt]{article}

% pdfLaTeX용 한글 패키지
\usepackage{kotex}

% 기본 패키지
\usepackage{amsmath, amsfonts, amssymb}
\usepackage{graphicx}
\usepackage{geometry}
\usepackage{fancyhdr}
\usepackage{enumitem}
\usepackage{array}
\usepackage{booktabs}
\usepackage{xcolor}
\usepackage{tikz}
\usepackage[colorlinks=true, urlcolor=blue, linkcolor=blue]{hyperref} % 하이퍼링크 (가장 마지막에)

% 페이지 여백 설정
\geometry{
  top=2.5cm, 
  bottom=2.5cm, 
  left=2cm, 
  right=2cm,
  headheight=20pt
}

% 머리글 및 바닥글 설정
\pagestyle{fancy}
\fancyhf{}
\renewcommand{\headrulewidth}{0.8pt}
\renewcommand{\footrulewidth}{0.4pt}
\lhead{\small 2025학년도 중간고사 대비 연습문제}
\chead{\small\textbf{컴퓨터 네트워크}}
\rhead{\small\thepage}
\cfoot{\scriptsize ※ 모든 문제의 답은 지정된 곳에 정확히 기입하시오.}

% 원형 숫자
\newcommand*\circled[1]{\tikz[baseline=(char.base)]{
  \node[shape=circle,draw,inner sep=0.7pt,minimum size=0.7em,line width=0.5pt] (char) {\small #1};}}

% 객관식 문항 스타일
\newlist{mchoice}{enumerate}{1}
\setlist[mchoice,1]{
  label=\protect\circled{\arabic*},
  leftmargin=2em,
  itemsep=0.3em,
  parsep=0pt
}

% 커스텀 명령어
\newcommand{\ansline}[1]{\underline{\hspace{#1}}}
\newcommand{\sectionbox}[1]{
  \vspace{0.5em}
  \noindent\fbox{\textbf{#1}}
  \vspace{0.5em}
}

\setlength{\parindent}{0pt}

\begin{document}

% 시험지 제목 블록
\begin{center}
  \Large\textbf{2025학년도 컴퓨터네트워크 중간고사 대비 연습문제지}\\[8pt]
  \large\textbf{Chapter 1. Introduction}\\[5pt]
  \small\textbf{contact: \href{mailto:wrim0923@ewhain.net}{wrim0923@ewhain.net} | github: \href{https://github.com/RainyForest23}{RainyForest23}}\\[5pt]
  \small\textcolor{gray}{Last updated: \today}
\end{center}

\vspace{0.5cm}

% 수험 정보
\begin{tabular}{|m{0.48\textwidth}|m{0.48\textwidth}|}
\hline
\textbf{성명:} \hfill & \textbf{학번:} \hfill \\[0.8em]
\hline
\end{tabular}
\vspace{1em}

\noindent
{\small ※ 각 문항의 정답을 해당란에 정확히 기입하시오. (총 30문항)}

\vspace{0.5em}
\noindent\rule{\textwidth}{0.8pt}
\vspace{0.5em}

% 문제 시작
\begin{enumerate}[itemsep=2em, leftmargin=2em, label={}]

\item[] \sectionbox{I. 용어 및 기본 개념 (5문제)}

\item[\textbf{1.}] 다음 중 인터넷 프로토콜의 역할에 대한 설명으로 가장 적절한 것은?
\begin{mchoice}
    \item 네트워크 엔티티 간에 전송 및 수신되는 메시지의 형식과 순서를 정의하고, 메시지 송수신 시 취해야 할 동작을 규정한다.
    \item 물리적 통신 링크의 전송 속도(대역폭)를 결정한다.
    \item 인터넷 표준 기관인 IETF의 물리적 위치를 지정한다.
    \item 종단 시스템(End system)을 라우터에 물리적으로 연결하는 방법을 정의한다.
\end{mchoice}

\item[\textbf{2.}] 종단 시스템(End System)을 지칭하는 다른 용어는 무엇이며, 이는 인터넷의 어디에서 네트워크 애플리케이션을 실행하는가?
\begin{enumerate}[label=\alph*., itemsep=0.3em, leftmargin=1.5em]
    \item 다른 용어: \ansline{3cm}
    \item 위치: 인터넷의 "\ansline{3cm}"
\end{enumerate}

\item[\textbf{3.}] "네트워크의 네트워크(network of networks)"라는 개념이 의미하는 것은 무엇인가?
\begin{enumerate}[label=\alph*., itemsep=0.3em, leftmargin=1.5em]
    \item 인터넷은 상호 연결된 \ansline{4cm}들로 구성되어 있음을 의미한다.
\end{enumerate}

\item[\textbf{4.}] 인터넷 표준 문서의 이름과, 해당 표준을 개발하는 기관의 약자를 각각 쓰시오.
\begin{enumerate}[label=\alph*., itemsep=0.3em, leftmargin=1.5em]
    \item 표준 문서 이름: \ansline{4cm}
    \item 표준 개발 기관 (3/4/5 계층): \ansline{4cm}
\end{enumerate}

\item[\textbf{5.}] 다음은 패킷 스위치(Packet Switch)의 종류를 설명한다. 빈칸에 알맞은 용어를 넣으시오.
\begin{enumerate}[label={}, itemsep=0.3em, leftmargin=1.5em]
    \item 패킷을 (데이터 청크) 전달하는 장치에는 \ansline{3cm}와(과) \ansline{3cm}가(이) 있다.
\end{enumerate}


\item[] \sectionbox{II. 네트워크 엣지 및 접속 네트워크 (5문제)}

\item[\textbf{6.}] 다음 중 액세스 네트워크(Access Network)에 대한 설명으로 거짓은? (\hspace{1cm})
\begin{enumerate}[label={}, itemsep=0.3em, leftmargin=1.5em]
    \item 액세스 네트워크는 종단 시스템(호스트)을 다른 먼 종단 시스템으로 가는 경로상의 첫 번째 라우터(엣지 라우터)에 물리적으로 연결하는 네트워크이다.
\end{enumerate}

\item[\textbf{7.}] 주거용 액세스 네트워크 기술 중, 케이블 TV 서비스가 제공되는 지역에 적용 가능하며 데이터와 TV가 공유 케이블 분배 네트워크를 통해 다른 주파수로 전송되는 기술은 무엇인가?
\begin{enumerate}[label=\alph*., itemsep=0.3em, leftmargin=1.5em]
    \item 기술명: \ansline{5cm}
\end{enumerate}

\item[\textbf{8.}] DSL (Digital Subscriber Line)과 HFC (Hybrid Fiber Coax)의 주요 차이점은 무엇인가?
\begin{mchoice}
    \item HFC는 중앙 오피스까지 전용 액세스를 제공하지만 DSL은 케이블 헤드엔드까지 공유 액세스를 사용한다.
    \item DSL은 기존 전화선을 사용하지만 HFC는 전용 광섬유 케이블만 사용한다.
    \item DSL은 중앙 오피스까지 전용 액세스 라인을 가지는 반면, HFC는 케이블 헤드엔드까지 액세스 네트워크를 여러 가정이 공유한다.
    \item DSL은 유선 통신만 지원하지만 HFC는 무선 통신만 지원한다.
\end{mchoice}

\item[\textbf{9.}] FTTH (Fiber To The Home) 기술의 두 가지 광 분배 네트워크(ODN) 아키텍처는 무엇인가?
\begin{enumerate}[label=\alph*., itemsep=0.3em, leftmargin=1.5em]
    \item \ansline{5cm} (스위치드 이더넷 방식)
    \item \ansline{5cm} (수동 분배기 방식)
\end{enumerate}

\item[\textbf{10.}] FWA (Fixed Wireless Access) 기술은 주로 어떤 네트워크 기술을 기반으로 하는가?
\begin{enumerate}[label=\alph*., itemsep=0.3em, leftmargin=1.5em]
    \item 기반 기술: \ansline{5cm}
\end{enumerate}


\item[] \sectionbox{III. 네트워크 코어 및 스위칭 (5문제)}

\item[\textbf{11.}] 서킷 스위칭(Circuit Switching) 방식의 주요 특징 세 가지를 설명하시오.
\begin{enumerate}[label=\alph*., itemsep=0.3em, leftmargin=1.5em]
    \item 통신 전에 자원 \ansline{4cm}이(가) 이루어진다.
    \item 자원이 통화 중 사용되지 않더라도 \ansline{4cm} 자원이다 (공유하지 않음).
    \item 일반적으로 \ansline{4cm} 성능을 보장한다.
\end{enumerate}

\item[\textbf{12.}] 패킷 스위칭(Packet Switching)이 서킷 스위칭보다 사용자들에게 네트워크 사용을 더 많이 허용할 수 있는 이유는 무엇인가?
\begin{enumerate}[label=\alph*., itemsep=0.3em, leftmargin=1.5em]
    \item \ansline{5cm}이(가) 가능하기 때문이다. (참고: 특히 버스티(bursty) 데이터 전송에 적합하다.)
\end{enumerate}

\item[\textbf{13.}] 패킷 스위칭 네트워크에서 라우터가 수행하는 두 가지 핵심 기능은 무엇인가?
\begin{enumerate}[label=\alph*., itemsep=0.3em, leftmargin=1.5em]
    \item \ansline{3cm}: 도착하는 패킷을 라우터의 입력 링크에서 적절한 출력 링크로 이동시키는 지역적(local) 동작.
    \item \ansline{3cm}: 패킷이 취할 송신지-수신지 경로를 결정하는 전역적(global) 동작.
\end{enumerate}

\item[\textbf{14.}] 패킷 스위칭에서 Store-and-forward (저장 후 전달) 지연이란 무엇인가?
\begin{enumerate}[label=\alph*., itemsep=0.3em, leftmargin=1.5em]
    \item \ansline{5cm}이(가) 다음 링크로 전송되기 전에 라우터에 도착해야 하는 데 걸리는 시간이다.
\end{enumerate}

\item[\textbf{15.}] Circuit Switching 망에서 TDM과 FDM을 비교하여 시간 슬롯 할당 방식의 차이점을 설명하시오.
\begin{enumerate}[label=\alph*., itemsep=0.3em, leftmargin=1.5em]
    \item FDM은 광학적/전자기적 \ansline{3cm}를(을) 좁은 대역으로 나누어 각 통화에 할당하는 반면, TDM은 \ansline{3cm}을(를) 슬롯으로 나누어 각 통화에 할당한다.
\end{enumerate}


\item[] \sectionbox{IV. 성능: 지연, 손실, 처리량 (5문제)}

\item[\textbf{16.}] 패킷이 라우터를 통과할 때 발생하는 네 가지 지연 요소를 나열하시오.
\begin{itemize}[itemsep=0.3em, leftmargin=1.5em]
    \item dproc: \ansline{3cm} delay
    \item dqueue: \ansline{3cm} delay
    \item dtrans: \ansline{3cm} delay
    \item dprop: \ansline{3cm} delay
\end{itemize}

\item[\textbf{17.}] $\text{d}_{\text{prop}}$ (전파 지연)과 $\text{d}_{\text{trans}}$ (전송 지연)을 정의하는 공식은 무엇인가?
\begin{enumerate}[label=\alph*., itemsep=0.3em, leftmargin=1.5em]
    \item $\text{d}_{\text{trans}}$ = \ansline{4cm} (단, L은 패킷 길이(bits), R은 링크 대역폭(bps))
    \item $\text{d}_{\text{prop}}$ = \ansline{4cm} (단, d는 물리적 링크 길이, s는 전파 속도)
\end{enumerate}

\item[\textbf{18.}] 패킷 큐잉 지연(Queueing delay)과 트래픽 강도(Traffic intensity, La/R) 사이의 관계를 설명하시오.
\begin{enumerate}[label=\alph*., itemsep=0.3em, leftmargin=1.5em]
    \item 트래픽 강도가 \ansline{4cm}에 가까워질수록 큐잉 지연은 급격히 증가한다.
\end{enumerate}

\item[\textbf{19.}] 패킷 손실(Packet Loss)은 어떻게 발생하는가?
\begin{enumerate}[label=\alph*., itemsep=0.3em, leftmargin=1.5em]
    \item 라우터 버퍼(큐)의 용량이 \ansline{4cm}이므로, 도착한 패킷이 꽉 찬 큐에 도달하면 드롭(손실)된다.
\end{enumerate}

\item[\textbf{20.}] 종단 간 처리량(End-end throughput)을 결정하는 요소는 무엇이며, 이를 무엇이라고 부르는가?
\begin{enumerate}[label=\alph*., itemsep=0.3em, leftmargin=1.5em]
    \item 요소: 종단 간 경로를 따라 흐름을 제한하는 링크.
    \item 명칭: \ansline{4cm} 링크.
\end{enumerate}


\item[] \sectionbox{V. 인터넷 구조 및 계층 (5문제)}

\item[\textbf{21.}] 가까운 거리에 위치하며 계층 수준이 비슷한 두 ISP가 상호 트래픽을 중개자 없이 직접 주고받기 위해 연결되는 방식을 무엇이라고 하는가?
\begin{enumerate}[label=\alph*., itemsep=0.3em, leftmargin=1.5em]
    \item 방식: \ansline{4cm}
    \item 이때 일반적으로 지불은 발생하는가? \ansline{2cm}
\end{enumerate}

\item[\textbf{22.}] 다수의 ISP들이 한 지점에서 만나 연결될 수 있도록 제3자 회사에서 스위치와 시설을 제공하는 독립적인 건물을 무엇이라고 하는가?
\begin{enumerate}[label=\alph*., itemsep=0.3em, leftmargin=1.5em]
    \item \ansline{5cm} (약자: IXP)
\end{enumerate}

\item[\textbf{23.}] 다른 상위 제공업체 ISP에 서비스 요금을 지불할 필요가 없는 최상위 ISP를 무엇이라고 부르는가?
\begin{enumerate}[label=\alph*., itemsep=0.3em, leftmargin=1.5em]
    \item \ansline{5cm} ISP
\end{enumerate}

\item[\textbf{24.}] Tier-1 ISP를 제외한 ISP가 두 개 이상의 Provider ISP에 연결하는 방식은 무엇이며, 이러한 방식의 장점은 무엇인가?
\begin{enumerate}[label=\alph*., itemsep=0.3em, leftmargin=1.5em]
    \item 방식: \ansline{5cm}
\end{enumerate}

\item[\textbf{25.}] 인터넷 프로토콜 스택 5계층을 나열하고, 각 계층의 프로토콜 예시를 하나씩 쓰시오.
\begin{center}
\renewcommand{\arraystretch}{1.3}
\begin{tabular}{|c|c|c|}
    \hline
    \textbf{계층} & \textbf{계층 이름} & \textbf{프로토콜 예시} \\
    \hline
    5 & Application & \ansline{3.5cm} \\
    \hline
    4 & Transport & \ansline{3.5cm} \\
    \hline
    3 & Network & \ansline{3.5cm} \\
    \hline
    2 & Link & \ansline{3.5cm} \\
    \hline
    1 & Physical & (Bits on the wire) \\
    \hline
\end{tabular}
\end{center}


\item[] \sectionbox{VI. 보안 및 기타 (5문제)}

\item[\textbf{26.}] 다음 중 Vint Cerf와 Robert Kahn이 제시한 인터넷워킹 원칙에 해당하지 않는 것은?
\begin{mchoice}
    \item 최소주의(Minimalism)와 자율성(Autonomy).
    \item 최선형 노력(Best effort) 서비스 모델.
    \item 라우터는 상태를 유지하지 않는다 (Stateless routers).
    \item 최종 사용자에게 엄격한 서비스 품질(QoS) 보장.
\end{mchoice}

\item[\textbf{27.}] DoS (Denial of Service) 공격의 목표는 무엇이며, 이 공격을 수행하기 위해 공격자들이 사용하는 세 가지 단계를 설명하시오.
\begin{enumerate}[label=\alph*., itemsep=0.3em, leftmargin=1.5em]
    \item 목표: 합법적인 트래픽이 자원을 사용할 수 없도록 \ansline{2.5cm} 또는 \ansline{2.5cm}를 위조 트래픽으로 압도하는 것.
    \item 3단계: 1. \ansline{2cm} 선택 2. 호스트 침투 3. 침해된 호스트에서 \ansline{2cm}(으)로 패킷 전송.
\end{enumerate}

\item[\textbf{28.}] 다른 프로그램에 기생하여 실행되며 사용자의 개입이 필요한 멀웨어(Malware) 유형은 무엇인가?
\begin{enumerate}[label=\alph*., itemsep=0.3em, leftmargin=1.5em]
    \item 멀웨어 유형: \ansline{4cm}
\end{enumerate}

\item[\textbf{29.}] IP 스푸핑(IP spoofing)이란 무엇인가?
\begin{enumerate}[label=\alph*., itemsep=0.3em, leftmargin=1.5em]
    \item \ansline{4cm} 주소를 위조한 패킷을 전송하는 행위이다.
\end{enumerate}

\item[\textbf{30.}] L-bit 크기의 패킷이 라우터에 들어올 때, 라우터의 포워딩 테이블을 참조하여 출력 포트를 결정하는 데 소요되는 지연은 무엇인가?
\begin{enumerate}[label=\alph*., itemsep=0.3em, leftmargin=1.5em]
    \item \ansline{5cm} delay.
\end{enumerate}

\end{enumerate}

\vfill
\begin{center}
  \rule{0.9\textwidth}{0.4pt}\\[8pt]
  {\small\textbf{--- 수고하셨습니다 ---}}\\[6pt]
  {\scriptsize ※ 본 문제지는 학습 목적으로 제작되었습니다.}
\end{center}

\end{document}