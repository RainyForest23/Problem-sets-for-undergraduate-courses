\documentclass[a4paper, 10pt]{article}

% pdfLaTeX용 한글 패키지
\usepackage{kotex}

% 기본 패키지
\usepackage{amsmath, amsfonts, amssymb}
\usepackage{graphicx}
\usepackage{geometry}
% \usepackage{multicol} % <-- 제거됨
\usepackage{fancyhdr}
\usepackage{enumitem}
\usepackage{array}
\usepackage{booktabs}
\usepackage{xcolor}
\usepackage{tikz}
\usepackage{hyperref}

% 페이지 여백 설정
\geometry{
  top=2.5cm, 
  bottom=2.5cm, 
  left=2cm, 
  right=2cm,
  headheight=20pt
}

% 머리글 및 바닥글 설정
\pagestyle{fancy}
\fancyhf{}
\renewcommand{\headrulewidth}{0.8pt}
\renewcommand{\footrulewidth}{0.4pt}
\lhead{\small 2025학년도 중간고사 대비 연습문제}
\chead{\small\textbf{컴퓨터 네트워크}}
\rhead{\small\thepage}
\cfoot{\scriptsize ※ 모든 문제의 답은 지정된 곳에 정확히 기입하시오.}

% 원형 숫자
\newcommand*\circled[1]{\tikz[baseline=(char.base)]{
  \node[shape=circle,draw,inner sep=0.7pt,minimum size=0.7em,line width=0.5pt] (char) {\small #1};}}

% 객관식 문항 스타일
\newlist{mchoice}{enumerate}{1}
\setlist[mchoice,1]{
  label=\protect\circled{\arabic*},
  leftmargin=2em,
  itemsep=0.3em,
  parsep=0pt
}

% 커스텀 명령어
\newcommand{\ansline}[1]{\underline{\hspace{#1}}}
\newcommand{\sectionsep}{
  \vspace{0.8em}
  \noindent\makebox[\linewidth]{\rule{0.9\linewidth}{0.4pt}}
  \vspace{0.8em}
}

% 섹션 타이틀 박스
\newcommand{\sectionbox}[1]{
  \vspace{0.5em}
  \noindent\fbox{\textbf{#1}}
  \vspace{0.5em}
}

\setlength{\parindent}{0pt}
% \setlength{\columnseprule}{0.4pt} % <-- 제거됨

\begin{document}


% 시험지 제목 블록
\begin{center}
  \Large\textbf{2025학년도 컴퓨터네트워크 중간고사 대비 연습문제지}\\[8pt]
  \large\textbf{Chapter 2. Application Layer}\\[5pt]
  \small\textbf{contact: \href{mailto:wrim0923@ewhain.net}{wrim0923@ewhain.net} | github: \href{https://github.com/RainyForest23}{RainyForest23}}\\[5pt]
  \small\textcolor{gray}{Last updated: \today}
\end{center}

\vspace{0.5cm}

% 수험 정보
\begin{tabular}{|m{0.48\textwidth}|m{0.48\textwidth}|}
\hline
\textbf{성명:} \hfill & \textbf{학번:} \hfill \\[0.8em]
\hline
\end{tabular}
\vspace{1em}

\noindent
{\small ※ 각 문항의 정답을 해당란에 정확히 기입하시오. (총 40문항, 100점)}

\vspace{0.5em}
\noindent\rule{\textwidth}{0.8pt}
\vspace{0.5em}


% 문제 시작
\begin{enumerate}[itemsep=3em, leftmargin=2em, label={}]

% ━━━━━━━━━━━━━━━━━━━━━━━━━━━━━━━━━━━━━━━━━
% I. 응용 계층의 기본 원리 및 구조
% ━━━━━━━━━━━━━━━━━━━━━━━━━━━━━━━━━━━━━━━━━

\item[] \sectionbox{I. 응용 계층의 기본 원리 및 구조}

\item[\textbf{1.}] \textbf{[TCP vs. UDP]} 다음 중 TCP와 UDP가 제공하는 서비스 모델에 대한 설명으로 \textbf{틀린} 것은?

\begin{mchoice}
  \item TCP는 신뢰할 수 있는 데이터 전송을 제공하며, UDP는 신뢰성을 보장하지 않는다.
  \item TCP는 연결 지향형으로 설정이 필요하지만, UDP는 연결 설정이 필요 없다.
  \item TCP는 흐름 제어와 혼잡 제어를 제공한다.
  \item UDP는 타이밍, 최소 처리량 보장, 보안 등 어떤 서비스도 제공하지 않지만, TCP는 최소 처리량 보장 기능을 제공한다.
  \item 실시간 스트리밍과 같은 지연에 민감한 애플리케이션은 종종 UDP를 선호한다.
\end{mchoice}

\item[\textbf{2.}] \textbf{[프로세스 식별]} 원격 호스트에서 실행되는 프로세스가 메시지를 수신하려면 반드시 식별자가 필요하다. 이 식별자는 호스트의 IP 주소와 무엇으로 구성되는가?

답: \ansline{4cm}

\item[\textbf{3.}] \textbf{[P2P]} P2P 구조는 새로운 피어가 합류할 때 새로운 서비스 수요뿐만 아니라 새로운 서비스 용량도 가져오기 때문에 스스로 확장성을 가진다. (\hspace{1cm})

\item[\textbf{4.}] \textbf{[P2P vs. 클라이언트-서버]} 파일 크기 $F$를 $N$개의 피어에게 분배할 때, P2P 아키텍처가 클라이언트-서버 아키텍처보다 효율적인 주요 이유와 \textbf{가장 거리가 먼} 것은?



\begin{mchoice}

\item P2P는 서버 외에 모든 피어의 업로드 용량을 활용하여 총 업로드 속도를 높일 수 있다.

\item 클라이언트-서버 방식은 서버가 $N$개의 파일 복사본을 보내야 하므로 분배 시간이 $N$에 비례한다.

\item P2P 방식은 서버 인프라에 대한 상당한 투자가 필요하지 않다.

\item P2P 방식은 피어의 수가 증가할수록 성능이 떨어진다.

\item P2P는 항상 켜져 있는 서버가 필요하지 않다.

\end{mchoice}



\item[\textbf{5.}] \textbf{[프로토콜 특성]} 서버가 과거 클라이언트 요청에 대한 정보를 유지하지 않는 프로토콜을 무엇이라고 하는가?



답: \ansline{4cm}



\item[\textbf{6.}] \textbf{[응용 계층 프로토콜]} 응용 계층 프로토콜이 정의하는 요소가 \textbf{아닌} 것은?



\begin{mchoice}

\item 교환되는 메시지 유형

\item 메시지 구문(syntax)

\item 메시지 의미론(semantics)

\item 메시지 전송 및 응답 규칙

\item 네트워크 코어 장치가 메시지를 라우팅하는 방법

\end{mchoice}



\item[\textbf{7.}] \textbf{[애플리케이션 아키텍처]} 다음 중 P2P 구조를 사용하는 예시는?



\begin{mchoice}

\item HTTP, SMTP

\item DNS, E-mail

\item BitTorrent, Skype

\item Web, File Transfer

\item Streaming video, Text messaging

\end{mchoice}





% ━━━━━━━━━━━━━━━━━━━━━━━━━━━━━━━━━━━━━━━━━

% II. 웹 및 HTTP

% ━━━━━━━━━━━━━━━━━━━━━━━━━━━━━━━━━━━━━━━━━



\item[] \sectionbox{II. 웹 및 HTTP}



\item[\textbf{8.}] \textbf{[HTTP 상태]} HTTP는 기본적으로 Stateless 프로토콜이지만, HTTP/2 및 HTTP/3에서는 Stateful 기능이 추가되었다. (\hspace{1cm})



\item[\textbf{9.}] \textbf{[HTTP 연결]} Non-persistent HTTP 연결에 대한 설명으로 \textbf{옳은} 것은?



\begin{mchoice}

\item 하나의 TCP 연결로 여러 객체를 전송할 수 있다.

\item 다수의 객체를 다운로드하려면 여러 TCP 연결이 필요하다.

\item 클라이언트가 참조된 객체를 발견하자마자 요청할 수 있다.

\item 모든 참조된 객체에 대해 RTT가 한 번만 소요된다.

\item 서버는 응답 후에도 연결을 열어 둔다.

\end{mchoice}



\item[\textbf{10.}] \textbf{[응답 시간]} 기본 HTML 파일이 10개의 작은 객체를 참조하고, 모두 동일 서버에 있다고 가정하자. Non-persistent HTTP에서 병렬 연결을 사용하지 않을 때, 모든 객체를 다운로드하는 데 필요한 총 RTT의 최소 횟수는? (DNS 조회 시간 제외)



답: \ansline{3cm} RTTs



\item[\textbf{11.}] \textbf{[응답 시간-병렬]} 문제 10과 동일한 상황에서, 브라우저가 10개의 병렬 TCP 연결을 열 때, 총 RTT의 최소 횟수는?



답: \ansline{3cm} RTTs



\item[\textbf{12.}] \textbf{[응답 시간-Persistent]} 문제 10과 동일한 상황에서, Persistent HTTP와 파이프라이닝을 사용할 때, 총 RTT의 최소 횟수는?



답: \ansline{3cm} RTTs



\item[\textbf{13.}] \textbf{[웹 캐시]} 웹 캐시를 운영했을 때의 이점으로 \textbf{가장 거리가 먼} 것은?



\begin{mchoice}

\item 클라이언트 요청에 대한 응답 시간을 줄인다.

\item 액세스 링크 트래픽을 줄여 비용을 절감한다.

\item 캐시가 없는 객체의 경우에도 지연이 줄어든다.

\item LAN 내부 트래픽이 줄어든다.

\item Conditional GET으로 전송 지연을 줄일 수 있다.

\end{mchoice}



\item[\textbf{14.}] \textbf{[HTTP 상태 코드]} Conditional GET 요청 시, 캐시된 사본이 최신 상태여서 서버가 객체를 전송하지 않을 때 응답하는 HTTP 상태 코드는?



답: \ansline{4cm}



\item[\textbf{15.}] \textbf{[HTTP 헤더]} 클라이언트가 이미 요청한 파일의 사본을 가지고 있는지 서버에게 알려주는 헤더는?



\begin{mchoice}

\item \texttt{User-Agent:}

\item \texttt{Accept-Language:}

\item \texttt{If-Modified-Since:}

\item \texttt{Connection:}

\item \texttt{Host:}

\end{mchoice}



\item[\textbf{16.}] \textbf{[HTTP/2]} HTTP/2는 객체를 프레임으로 나누어 전송 순서를 섞음으로써 HOL Blocking 문제를 완화한다. (\hspace{1cm})




% ━━━━━━━━━━━━━━━━━━━━━━━━━━━━━━━━━━━━━━━━━

% III. 전자 메일

% ━━━━━━━━━━━━━━━━━━━━━━━━━━━━━━━━━━━━━━━━━



\item[] \sectionbox{III. 전자 메일 (E-mail)}



\item[\textbf{17.}] \textbf{[SMTP vs. HTTP]} SMTP와 HTTP는 모두 TCP를 사용하지만, SMTP는 데이터를 밀어 넣는(push) 프로토콜이고, HTTP는 당겨오는(pull) 프로토콜이다. (\hspace{1cm})



\item[\textbf{18.}] \textbf{[SMTP 특성]} SMTP의 특징으로 \textbf{옳지 않은} 것은?



\begin{mchoice}

\item 기본적으로 TCP 포트 25를 사용한다.

\item 전송 서버에서 수신 서버로 직접 메일을 전송한다.

\item 메시지는 반드시 7비트 ASCII로 표현되어야 한다.

\item 영구적 TCP 연결을 사용한다.

\item 하나의 메시지에 여러 객체를 담을 수 없으며, 각 객체는 별도로 전송되어야 한다.

\end{mchoice}



\item[\textbf{19.}] \textbf{[이메일 프로토콜]} Bob이 메일 서버의 받은 편지함에서 메일을 가져와 읽을 때 사용할 수 있는 프로토콜은?



\begin{mchoice}

\item SMTP

\item HTTP (웹 기반이 아닌 경우)

\item IMAP

\item a와 c 모두

\item b와 c 모두

\end{mchoice}



\item[\textbf{20.}] \textbf{[이메일 전송]} Alice의 사용자 에이전트가 Bob의 메일 서버와 직접 통신하지 않고, Alice의 메일 서버를 경유하도록 설계된 이유로 \textbf{가장 적절한} 것은?



\begin{mchoice}

\item Alice의 UA가 TCP 연결에 실패할 경우 메일 서버가 재시도할 수 있다.

\item Alice의 메일 서버가 DNS resolution을 대신 처리한다.

\item Alice의 메일 서버가 여러 사용자의 메일을 동시에 효율적으로 전송할 수 있다.

\item 위의 모두

\end{mchoice}



\item[\textbf{21.}] \textbf{[SMTP 상태]} SMTP는 핸드셰이킹, 메시지 전송, 종료의 세 단계를 거치므로 stateful 프로토콜이다. (\hspace{1cm})






% ━━━━━━━━━━━━━━━━━━━━━━━━━━━━━━━━━━━━━━━━━

% IV. DNS

% ━━━━━━━━━━━━━━━━━━━━━━━━━━━━━━━━━━━━━━━━━



\item[] \sectionbox{IV. DNS}



\item[\textbf{22.}] \textbf{[DNS 계층]} DNS는 핵심 인터넷 기능이지만, 네트워크 코어가 아닌 최종 시스템에서 실행되는 응용 계층 프로토콜이다. (\hspace{1cm})



\item[\textbf{23.}] \textbf{[DNS 서비스]} DNS가 제공하는 주요 서비스가 \textbf{아닌} 것은?



\begin{mchoice}

\item 호스트 이름-IP 주소 변환

\item 호스트 별명을 정식 이름으로 변환

\item 메일 서버 별명

\item 신뢰할 수 있는 데이터 전송

\item 부하 분산

\end{mchoice}



\item[\textbf{24.}] \textbf{[DNS 쿼리]} DNS 쿼리 요청을 받은 서버가 다른 DNS 서버에게 추가 쿼리를 발행하고 최종 결과를 클라이언트에게 응답하는 방식은?



\begin{mchoice}

\item 캐싱 쿼리

\item 반복 쿼리

\item 재귀 쿼리

\item 권한 쿼리

\end{mchoice}



\item[\textbf{25.}] \textbf{[DNS 오버헤드]} 재귀적 쿼리가 반복적 쿼리보다 상위 계층 DNS 서버에 더 큰 오버헤드를 부과하는 이유는?



\begin{mchoice}

\item 클라이언트가 여러 서버에 개별적으로 쿼리를 보내야 하므로

\item 상위 서버가 더 넓은 지역의 LDNS를 대상으로 더 많은 쿼리를 처리해야 하므로

\item 재귀적 쿼리는 항상 캐싱을 무시하므로

\item 재귀적 쿼리는 DNS 영역 전송 시에만 사용되므로

\end{mchoice}



\item[\textbf{26.}] \textbf{[LDNS]} 로컬 DNS 서버는 DNS 계층 구조에 엄격하게 속하지 않지만, 캐시를 보유하고 프록시 역할을 수행하여 DNS resolution 속도를 향상시킬 수 있다. (\hspace{1cm})



\item[\textbf{27.}] \textbf{[DNS 레코드]} 별칭이 어떤 정식 이름의 별명인지 알려주는 DNS 레코드 유형은?



\begin{mchoice}

\item A 레코드

\item NS 레코드

\item CNAME 레코드

\item MX 레코드

\end{mchoice}






% ━━━━━━━━━━━━━━━━━━━━━━━━━━━━━━━━━━━━━━━━━

% V. 비디오 스트리밍, CDN 및 소켓 프로그래밍

% ━━━━━━━━━━━━━━━━━━━━━━━━━━━━━━━━━━━━━━━━━



\item[] \sectionbox{V. 비디오 스트리밍, CDN 및 소켓}



\item[\textbf{28.}] \textbf{[CDN]} CDN 서버 클러스터를 많은 액세스 네트워크 깊숙이 배치하여 지연을 줄이는 방식은?



\begin{mchoice}

\item Bring Home

\item Enter Deep

\item Mega Server

\item Cloud Deployment

\end{mchoice}



\item[\textbf{29.}] \textbf{[DASH]} DASH 프로토콜에 대한 설명으로 \textbf{틀린} 것은?



\begin{mchoice}

\item 서버는 비디오를 여러 청크로 나누고, 각 청크를 다양한 비트레이트로 저장한다.

\item 매니페스트 파일은 청크에 대한 URL 정보를 제공한다.

\item 클라이언트는 주기적으로 대역폭을 측정하여 최대 코딩 속도를 선택한다.

\item TCP 연결을 맺고 있는 DASH 서버는 항상 동일하다.

\item 비디오 스트리밍의 지능이 주로 클라이언트 측에 있다.

\end{mchoice}



\item[\textbf{30.}] \textbf{[TCP 소켓]} TCP 서버가 N개의 클라이언트와 동시에 연결을 지원할 때, 서버가 필요로 하는 최소 소켓 개수는?



답: \ansline{3cm} 개



\item[\textbf{31.}] \textbf{[UDP 소켓]} UDP 서버가 N개의 클라이언트로부터 동시에 데이터그램을 수신할 때, 서버가 필요로 하는 최소 소켓 개수는?



답: \ansline{3cm} 개



\item[\textbf{32.}] \textbf{[소켓-TCP]} TCP를 사용하는 클라이언트-서버 애플리케이션의 경우, 서버 프로그램은 클라이언트 프로그램보다 먼저 실행되어야 한다. (\hspace{1cm})



\item[\textbf{33.}] \textbf{[소켓-UDP]} UDP를 사용하는 클라이언트-서버 애플리케이션의 경우, 클라이언트 프로그램은 서버 프로그램보다 먼저 실행될 수 있다. (\hspace{1cm})



\item[\textbf{34.}] \textbf{[웹 캐시 계산]} 기관 네트워크에서 평균 객체 크기가 900,000비트이고, 외부 서버로의 평균 요청률이 초당 15회이다. 히트율 0.4인 웹 캐시를 설치하면, 액세스 링크 외부로 나가는 요청률 $\beta'$는 초당 몇 회로 감소하는가?



답: \ansline{3cm} request/sec



\item[\textbf{35.}] \textbf{[캐시 지연]} 문제 34의 상황에서, 히트율 0.4인 웹 캐시 설치 후의 평균 액세스 지연(캐시 미스 시)을 계산하시오. (소수점 셋째 자리)



답: \ansline{3cm} seconds



\item[\textbf{36.}] \textbf{[총 응답 시간]} 문제 34의 상황에서, 캐시 설치 후의 총 평균 응답 시간은? (외부 인터넷 지연 2초, LAN 지연 무시, 소수점 셋째 자리)



답: \ansline{3cm} seconds




% ━━━━━━━━━━━━━━━━━━━━━━━━━━━━━━━━━━━━━━━━━

% VI. 종합 및 비교

% ━━━━━━━━━━━━━━━━━━━━━━━━━━━━━━━━━━━━━━━━━



\item[] \sectionbox{VI. 종합 및 비교}



\item[\textbf{37.}] \textbf{[TCP 연결]} 두 호스트 사이에 TCP 연결을 설정한다는 의미는 경유하는 중간 라우터에 양쪽 TCP 정보를 기록한다는 뜻이다. (\hspace{1cm})



\item[\textbf{38.}] \textbf{[데이터 인코딩]} SMTP는 이메일 메시지 본문을 7-bit ASCII로 표현해야 하므로, 바이너리 데이터를 첨부하기 위해서는 전송 전에 \ansline{4cm} 과정을 거쳐야 한다.



\item[\textbf{39.}] \textbf{[클라이언트-서버]} 클라이언트-서버 구조에서 서버는 영구적인 IP 주소와 잘 알려진 포트 번호를 가져야 하지만, 클라이언트는 유동적인 IP 주소를 가져도 무방하다. (\hspace{1cm})

\item[\textbf{40.}] \textbf{[DNS-HTTP 연동]} Alice가 www.nu.com을 처음 방문하기 위한 과정을 시간 순서대로 나열하시오.

\begin{enumerate}[label=\arabic*., itemsep=0.3em, leftmargin=1.5em, parsep=0pt]
  \item .com TLD 서버는 NU의 authoritative DNS 서버 정보를 Alice의 LDNS에게 보낸다.
  \item NU의 authoritative DNS 서버가 www.nu.com의 IP주소를 Alice의 LDNS에게 보낸다.
  \item Alice의 호스트가 www.nu.com의 IPv4 주소를 묻는 DNS query를 LDNS에게 보낸다.
  \item Alice의 LDNS가 NU의 authoritative DNS 서버에게 DNS query를 보낸다.
  \item Alice의 browser가 웹 서버로 TCP 연결을 맺고 HTTP request를 전송한다.
  \item Alice의 LDNS가 (www.nu.com, 21.21.21.4, A)를 포함한 DNS reply를 Alice에게 전달한다.
  \item Alice의 LDNS는 DNS query를 .com TLD DNS 서버에게 보낸다.
\end{enumerate}

\vspace{0.5em}
\noindent
답: 3 → \ansline{0.8cm} → \ansline{0.8cm} → \ansline{0.8cm} → \ansline{0.8cm} → \ansline{0.8cm} → \ansline{0.8cm}


\end{enumerate}


\vfill

\begin{center}
  \rule{0.9\textwidth}{0.4pt}\\[8pt]
  {\small\textbf{--- 수고하셨습니다 ---}}\\[6pt]
  {\scriptsize ※ 본 문제지는 학습 목적으로 제작되었습니다.}
\end{center}

\end{document}